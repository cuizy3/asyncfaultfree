%4/4 changes, made general edits and changed (s_i,s_j) \in T to T(s_i,s_j)
%changed all the S_w to S, w_i w_j to s_i s_j, w to v, k l to v w
%4/10 incorporated new sugestions, also what 103 suggested

\documentclass{article}
\usepackage{amsfonts}
\usepackage[utf8]{inputenc}
\usepackage[T1]{fontenc}
\usepackage{lmodern}

\title{Proof for normal operation of the duplicated scheme for asynchronous circuits}
\author{Zoey Zhou}
\date{\today}  
\usepackage{graphicx}
\graphicspath{ {c:/Desktop/} }

\usepackage{amsthm}
\newtheorem*{definition}{Definition}
\newtheorem*{claim}{Claim}
\newtheorem*{theorem}{Theorem}
 
\begin{document}
\section{Normal operation analysis}
Definitions:  
\begin{definition}A state graph consists of <V, S, T> where $V$ is the set of finite signals $v_1 .. v_n$.  $S$ is the function $S: V \to \{0,1\}$ %<-this might not be right math... and s(k) denotes the (binary) value of the state at signal $v_k$.  
which maps each variable to a boolean, an element of $S$ is called a state. $T$ is the set of all transitions in the state graph and $T \subseteq S \times S$.  \end{definition}

A transition has the additional constraint that only one signal is allowed to change between the two states in the transition.
\begin{definition}A transition $T(s_i, s_j)$ denotes a transition from the state $s_i$ to $s_j$.  Then there exists a signal $w$ and any signal $u$ where $u \neq w$ such that $s_i(w)\neq s_j(w)$ and $s_i(u)=s_j(u)$ . 
\end{definition}

%major revision needed
State graphs can be implemented as an asynchronous circuit where the signals of a state graph $V$ maps one-to-one to wires in the circuit. %(one to one mapping?)  
Then the set of states $S$ can also describe the state of wires of the circuit.  The circuit has a set of transitions.  A transition of the circuit from a state $s_i \in S$, to a state $s_j \in S$ exists if and only if $T(s_i, s_j)$.%set of transitions instead of transitions
\begin{definition}A wire $u$ in the circuit is \textbf{excited} in state $s_i$ if $T(s_i,s_j)$ and $s_i(u) \neq s_j(u)$.  Alternatively if one treats the output of wire $u$ as a function of the current state $s_i$, then $f(s_i)\neq s_i(u)$\end{definition}
%(may need to change this definition to s'=/=s) <- added in

Semi-modularity means once a wire is excited, it stays excited until it transitions to the excited value.  Circuits that are semi-modular are also speed independent.  Formally this can be defined as follows:
\begin{definition}An asynchronous circuit is \textbf{semi-modular} if for every $s_i$ and $s_j$ such that $T(s_i,s_j)$ holds, and let $u$ and $w$ be wires such that $s_i(u) \neq s_j(u)$ and $w\neq u$, then if $s_i(w)$ is excited, $s_j(w)$ is also excited \end{definition}

\begin{definition} \textbf{Trace} $\sigma$ of a circuit is a sequence of states of infinite length where $\sigma_i$ is the $i^{th}$ term of the sequence.  $\sigma: \mathbb{N} \to S$ and $T(\sigma_i,\sigma_{i+1})$ for $\forall i \in \mathbb{N}$\end{definition} 
% all states or just the state graph ones

\begin{definition} Two circuits A and B are \textbf{equivalent} if the set of all traces from circuit A and circuit B are equal $\{\sigma^A |\sigma^A$ is a possible trace in $A\}=\{\sigma^B |\sigma^B$ is a possible trace in $B\}$ (Traces(A)= Traces(B)?).
\end{definition}
%good enough for now, do I need if and only if?

An additional property of the asynchronous circuit is that one can view each wire as a separate black box.  For a wire $w$, inputs from the rest of the circuit ($I_w$ and $I_w \subseteq$ V) feed into a gate and outputs the wire value $w$.  We call the instantaneous value of the gate for $w$ of a state $s$ as $f_w(s|_I)$, where f is the function $f:[I \to \{0,1\}] \to \{0,1\}$ and $s|_I$ is the state $s$ projected on the input variables $I_w$.  If inputs to the gate remain constant, the output $w$ will eventually take the value specified by $f_w(s|_I)$  %maybe change w again, also the f(s) vs f(I) thing...
\newline

We transform the circuit by making a duplicated copy of the gates to each wire, and connect the outputs of the duplicated gates to two c-elements.  We label the duplicated circuit halves as circuit A and circuit B.  Then for each wire $w$ we add in intermediate wires $x^A$ and $x^B$ to label the output of the gate of $w$ for circuits A and B respectively.  $x^A$ and $x^B$ are also the inputs to two C-elements.  We can (arbitrarily) label one of the C-element outputs as $w^A$ and the other output as $w^B$.  These then connect to other gates in circuit A and circuit B respectively.  (Note that the outputs of the C-elements can be arbitrarily named and connected to the next gates, and the next gates can also be arbitrarily named $A$ or $B$ as long as it does not create any inconsistencies in the naming convention) %possibly expand on this and also show why such a naming can always occur?
\newline
\includegraphics[width=\textwidth]{circuitforproof}
Because the transformed circuit has some extra wires, we want to define a correspondence to be able to compare between the wires of the original circuit and that of the duplicated circuit halves.  For the duplicated circuit, a combined state is the pairing of the state of circuit A and the state of circuit B. It has quadruple the number of wires for each wire in the original circuit.  The state graph for a duplicated circuit is defined on this combined state.  We define the mapping $h^A: s^{full} \to s^A$ where $s^{full}$ is the combined state of a duplicated circuit and $s^A$ are made up of the wires $w^A$ that correspond to the original wires $w$.  $h^B$ is similarly defined to map to $s^B$.  Note that the $h(\sigma)$ mappings are overloaded for a trace, where it outputs a mapping for each of the states in a trace.%not sure if I want the mapping to be from variable to variable or specific state to state
For a trace in the duplicated circuit we project the full states onto a shortened list of wires (ie the wires corresponding to the original circuit). The resulting trace may have successive repeated states and we delete the repeats to obtain a valid trace on the shortened wires.  %do I need to define all the new transitions in duplicated circuit?  maybe
\newline
We want to show this circuit behaves similar to the original circuit under normal operation.  In other words we want to prove that after applying either mapping of $h^A$ or $h^B$, the resultant circuit and the original circuit are equivalent.  
\begin{theorem}Starting from all valid initial states in the original circuit (and a predefined correspondence to the initial state in the duplicated circuit) and for all $n$, if $\sigma$ is a trace in the duplicated circuit and $\sigma_n$ is the sequence of the first n states of $\sigma$, then $h^A(\sigma_n)$ is a prefix of a trace in the original circuit  (h might make the trace shorter, so the prefix will not be of length n).  
\end{theorem}
\begin{proof}
We prove this by induction on an expanded list of properties, and then show that these properties prove the original theorem is true.  Without loss of generality we assume these properties hold for only the A half of the circuit.  The same reasoning then applies to the B half of the circuit.  We define an induction hypothesis which is the conjunction of these three properties for some $\sigma$ that is a trace in the duplicated circuit and $n \in \mathbb{N}$:  
%We assume the following properties for $\sigma_i$ for some $i \in \mathbb{N}$:
\begin{enumerate} %1) 2) 3)
\item   $h^A(\sigma_n)$ is a prefix of a trace in the original circuit.  We define $s_n$ as the $n^{th}$ element of $\sigma_n$ (the current state). %(would be two steps ahead.... don't even think I can show it with the other properties...)If a transition occurs from $s_n$, then the next state $s_{n+1}$ is either $s_n=s_{n+1}$ or $T(s_n,s_{n+1})$  %is s_n a state in the original??  or duplicated with all 
%(either trace or state)...  Also maybe delete the trace line... it becomes a general provable property
%I feel like this is a bit funky, may have to redo 
\item  (If a wire is excited in the original circuit then exactly one of the gate or the c-element is excited in the duplicated half.  If a wire is not excited, then neither the gates or the c-element is excited) If the duplicated circuit is in state $s_n$ and the original circuit is in state $h^A(s_n)$, and for all wires $w$ that are excited in this state $f_w(s_n|_I)\neq w$ in original circuit, then $f_w(s_n|_I) \neq x^A$ or $x^A \neq w^A$ in duplicated circuit (exactly one of these is true).  
In addition, if wire $w$ is not excited in the original circuit $f_w(s_n|_I)=w$, then in the duplicated circuit half $f_w(s_n|_I)=x^A=w^A$.  %<define correspondence> <- defined above
\item  For all output wires $w$ in state $s_n$, if the corresponding output wire in $A$ and $B$ are different, this indicates that one of the c-elements to $A$ or $B$ is excited.  For all output wires $w$, if $w^A\neq w^B$ then $x^A=x^B$ 
\end{enumerate}

The base case is constructed as follows:  given any initial state in the original circuit $s \in S$, for each wire $w$ assign the values for the wires in the duplicated circuit according to $x^A=x^B=w^A=w^B=w$.  This is $s_0$ in the duplicated circuit.  It is simple to see that this initial state satisfies all three properties.  %<need a formal def of similarly<- done <- need to be augmented cuz additional wires <- fine if I talk about deleting extra wires and removing 
%Of course this depends on the initial configuration of the duplicated circuit.  But if we set up the initial correspondence correctly, we can show certain properties in all of the traces that can occur that will lead to the above conclusion.
\newline
%need to talk about half of a duplicated circuit <- done 
Next is the induction step, assume we are in a full state $s_k=(s_i^A$ $s_j^B)$ and these three properties are true, we will prove each of the properties are true for $k+1$. 
\begin{enumerate} %prove the same properties for the next step
\item  Due to property 2) we know that only wires excited in the original circuit in state $s_i$ may have excited components.  For all wires $w$ that are excited in the original circuit, if logic gate is excited $f_w(s_i|_I)\neq x^A$ then a transition on $x^A$ may occur.  The value of $w^A$ remains the same and h^A(s_k)=h^A(s_{k+1}).  Otherwise if $x^A \neq w^A$ then there are two cases, if $x^A = x^B$ and if $x^A \neq x^B$. Case 1: $x^A = x^B$ the c-element is excited and $w^A$ can transition to the new value $w_{next}^A=x^A$.  Case 2: $x^A \neq x^B$ and no transition can occur. The first case is the same as $w$ transitioning in the original circuit and $T(h^A(s_k), h^A(s_{k+1}))$.  Thus property 1 is true for $n+1$.
\item  A transition may only take place in circuit A if the wire $w$ is excited in the original.   If $f_w
(s_k)\neq x^A$ and a transition on $x^A$ occurs then $f_w(s_{k+1}|_I)= x_{next}^A$ and $x_{next}^A\neq w_{next}^A$.  
If $x^A \neq w^A$ and $x^A = x^B$ and $w^A$ transitions then  $x_{next}^A= w_{next}^A$, in addition since $h^A(s_{k+1})$ is also a state in the original circuit (from above) then the set of newly excited wires are the same in the original circuit and circuit A.  
%need to expand on newly excited wires
%do I need to define transitions as it occurs here?
For all wires w in the newly excited set, since w was previously not excited, then $f_w(s_{k+1})\neq x_{next}^A$ and $x_{next}^A= w_{next}^A$.  Due to semi-modularity the wires excited at $s_k$ is still excited at $s_{k+1}$ in the original circuit with the exception of the wire that transitioned.  In circuit A, if $f_w(s_k|_I)\neq x^A$ then $f_w(s_{k+1}|_I)\neq x_{next}^A$ since it is the same logic gates and inputs as the original circuit.  And if $x^A \neq w^A$ then $x_{next}^A \neq w_{next}^A$ since no transitions occured in $x^A$ or $w^A$. %what happens in B does not matter here
For a wire $w$ not excited in $s_k$ and it is not excited in $s_{k+1}$ in the original circuit, in circuit A we have $x^A=w^A$ then $x_{next}^A=w_{next}^A$ and $f_w(s_{k+1}|_I)= x_{next}^A$.  Thus property 2 holds in any next state $s_{k+1}$
\item  We will prove this property in $n+1$ by contradiction. Suppose this property is not true in a possible next state so $w_{next}^A\neq w_{next}^B$ and $x_{next}^A\neq x_{next}^B$.  This means that in the current state if $w^A \neq w^B$ and $x^A=x^B$ then a transition on $x^A$ or $x^B$ occurs.  Without loss of generality, assume $w^B=x^B$, then $x^A$ cannot transition or it would violate property 2 above.  Then $x^B$ transitions.  Because of assumption of 3 in state $s_i^A$, $s_j^B$ there is a possible sequence of transitions on the inputs of wire w in circuits A and B so that they are equal.  This means following a trace of h steps in the original circuit  $x_{k+h}^A$ can be excited which again violates property 2.  Thus the initial assumption is false and this property is true in all possible next states.  Therefore property 3 is true for $k+1$.
\end{enumerate}
Finally there is always a transition available since if $w^A\neq w^B$ we can always use the trick above to produce a sequence of transitions so that $w^A= w^B$.  And if the circuit is in this state there is a next availble transition following the original circuit.  Thus the three properties are true for the step $k+1$.  Induction follows. Property 1 is the original theorem which must also be true.%dunno about the i and n notation... then add in why (below)
\end{proof}
%with these 3 properties show the original claim

\end{document}

