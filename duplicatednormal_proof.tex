\documentclass{article}
\usepackage[utf8]{inputenc}
\usepackage[T1]{fontenc}
\usepackage{lmodern}

\title{Proof for normal operation of the duplicated scheme for asynchronous circuits}
\author{Zoey Zhou}
\date{\today}  
\usepackage{graphicx}
\graphicspath{ {c:/Users/cuizy/Downloads/} }

\usepackage{amsthm}
\newtheorem*{definition}{Definition}
\newtheorem*{claim}{Claim}
 
\begin{document}
\section{Normal operation analysis}
Definitions:  
\begin{definition}A state graph consists of <V, S, T> where $V$ is the ordered set of finite signals where $1\leq k \leq n$ indexes into $V$ and gives the signal $v_k$.  $S$ is the set of all states in the state graph $V \mapsto B$  $B={0,1}$(note that this is a subset of $2^n$) %<-this might not be right math...
and s(k) denotes the (binary) value of the state at signal $v_k$.  T is the set of all transitions in the state graph $T \subseteq S X S$ and $(s_i, s_j) \in T$ denotes a transition from the state $s_i$ to $s_j$.  A transition pair $s_i$ and $s_j$ are under additional constraint 
$\exists k$ st $s_i(k)\neq s_j(k)$ and $s_i(l)=s_j(l)$ for all other $l \neq k$. \end{definition}

State graphs can be implemented as an asynchronous circuit where the signals of a state graph $s$ maps to wires in the circuit $w$. %(one to one mapping?)  
We can also map the set $S$ to a set $S_w$ of state of wires. The circuit has a set of transitions, for all $w_i \in S_w$, $(w_i, w_j)$ is a transition of the circuit if and only if $(s_i, s_j) \in T$.
\begin{definition}A wire $k$ in the circuit is \textbf{excited} in state $w_i$ if $(w_i,w_j)$ is a transition and $w_i(k) \neq w_j(k)$.  Alternatively if one treats the output of wire $k$ as a function of the current state $w_i$, then $f(w_i)\neq w_i(k)$\end{definition} 
%(may need to change this definition to s'=/=s) <- added in

\begin{definition}An asynchronous circuit is \textbf{semi-modular} if $\forall w_i \in S_w$ and $\forall (w_i,w_j)$ transitions and $w_i(k) \neq w_j(k)$, and $l\neq k$ then if $w_i(l)$ is excited, $w_j(l)$ is also excited \end{definition}
Semi-modularity means a wire stays excited until it transitions to the excited value.  Circuits that are semi-modular are also speed independent. \newline

\begin{definition} \textbf{Trace} of a circuit is a sequence of states of length n.  $[0 .. n-1] \mapsto S$ and $(s_i,s_j) \in T$ for $0 \leq i \leq n-2$\end{definition} 
% all states or just the state graph ones

\begin{definition} Two circuits A and B are \textbf{similar} if all traces from A can occur in B and vice versa\end{definition}
%good enough for now

An additional property of the asynchronous circuit is that one can view each wire as separate black boxes.  Inputs from the rest of the circuit (other $w_k$) feed into a logic block and outputs the wire value $w_l$.  We call the instantaneous value of the logic block $l$ of a state s as $f_l(s)$.
\newline

We transform the circuit by making a duplicated copy of the logic blocks to each wire, and interconnecting the outputs of the logic blocks to two c-elements.  We label the duplicated circuit as circuit A and circuit B.  Then we add in intermediate wires $w^A'(k)$ and $w^B'(k)$ that is the output
of the logic blocks $k$ of circuits A and B respectively.  $w^A'(k)$ and $w^B'(k)$ are inputs to two C-elements.  We can (arbitrarily) label one of the C-element outputs as $w^A(k)$ and the other output as $w^B(k)$.  These then connect to other circuit A and circuit B logic blocks respectively.
We want to show this circuit behaves similarly to the original circuit under normal operation.  %<need a formal def of similarly<- done <- need to be augmented cuz additional wires
Of course this depends on the initial configuration of the transformed circuit.  But if we set up the initial correspondence correctly, we can show certain properties in all of the traces that can occur.  
\newline
%need to talk about half of a duplicated circuit <- done

To show that the transformed circuit behaves similar to the original circuit, we have the following properties:
\begin{enumerate} %1) 2) 3)
\item  the trace that occurs in A and B is a trace in original circuit.  If a transition happens from $s_i$, then the next state $s_j$ is either $s_i=s_j$ or $(s_i,s_j) \in T$  %(either trace or state)...
%I feel like this is a bit funky, may have to redo 
\item  Given a state w in the original circuit, if wire k is excited in this state, $f_k(w)\neq w(k)$, then if half of the duplicated circuit A or B is also in this state then $f_k(w) \neq w'(k)$ or $w'(k) \neq w(k)$ (exactly one of these is true).  
In addition, if wire k is not excited in the original circuit, $f_k(w)=w(k)$, then in the duplicated circuit half $f_k(w)=w'(k)=w(k)$.  %<define correspondence>
\item  For some k, if $w^A(k)\neq w^B(k)$ then $w^A'(k)=w^B'(k)$ (if the same output wire in A and B are different, this indicates that one of the c-elements to A or B is excited.)
\end{enumerate}
We prove these properties through induction.  Assume we are in a state \{$w_i^A$ $w_j^B$\} where these 3 properties are true.  Then 
\begin{enumerate} %prove the same properties for the next step
\item  Due to property 2) we know that only wires excited in the original circuit (in state $w_i^A$ or $w_j^B$) may have excited components.  We focus on $w_i^A$ first, for all wires k that are excited in the original, 
if $f_k(w_i^A)\neq w_i^A'(k)$ then a transition on $w_i^A'(k)$ may occur.  Otherwise if $w_i^A'(k) \neq w_i^A(k)$ then if $w_i^A'(k) = w_j^B'(k)$ the c-element is excited and $w_i^A(k)$ can transition to the new value $w_{i+1}^A(k)=w_i^A'(k)$. 
Since k is excited in the original circuit, $w_{i+1}^A$ is also a state in the original circuit and $(w_
i^A, w_{i+1}^A)\in T$.  Otherwise the c-element is not excited and no transition can occur. The same is true for $w_j^B$
\item  Using the same reasoning as above a transition may only take place in circuit A if the wire k is excited in the original.   if $f_k(w_i^A)\neq w_i^A'(k)$ and a transition on $w_i^A'(k)$ occurs then $f_k(w_{i+1}^A)= w_{i+1}^A'(k)$ and $w_{i+1}^A'(k)\neq w_{i+1}^A(k)$.  
If $w_i^A'(k) \neq w_i^A(k)$ and $w_i^A'(k) = w_j^B'(k)$ and $w_i^A(k)$ transitions then  $w_{i+1}^A'(k)= w_{i+1}^A(k)$, in addition since $w_{i+1}^A$ is also a state in the original circuit (from above) then the set of newly excited wires are the same in the original circuit and circuit A.  
%need to expand on newly excited wires
For all wires k in the newly excited set, since k was previously not excited, then $f_k(w_{i+1}^A)\neq w_{i+1}^A'(k)$ and $w_{i+1}^A'(k)= w_{i+1}^A(k)$.  Due to semi-modularity the wires excited at $w_(i)^A$ is still excited at $w_{i+1}^A$ with the exception of the wire that transitioned and the same is true in circuit A.
And the wires not excited at $w_(i)^A$ is not excited at $w_{i+1}^A$ and the same is true in circuit A.  Thus property 2 holds in any next state $w_{i+1}^A$
\item  Suppose this property is not true in a possible next state ($w_{i+1}^A(k)\neq w_{j+1}^B(k)$ and $w_{i+1}^A'(k)\neq w_{j+1}^B'(k)$).  This means that in the current state if $w_i^A(k)\neq w_j^B(k)$ $w_i^A'(k)=w_j^B'(k)$ then a transition on $w_i^A'(k)$ or $w_j^B'(k)$ occurs.  Without loss of generality, assume $w_j^B(k)=w_j^B'(k)$, then $w_i^A'(k)$ cannot transition or it would violate property 2 above.  Then $w_j^B'(k)$ transitions.  Because of assumption of 3 in state $w_i^A$, $w_j^B$ there is a possible sequence of transitions on the inputs of wire k in circuits A and B so that they are equal.  This means following a trace of h steps in the original circuit  $w_{i+h}^A'(k)$ can be excited which again violates property 2.  Thus the initial assumption is false and this property is true in all possible next states 
\end{enumerate}
Finally there is always a transition available since if $w_i^A(k)\neq w_j^B(k)$ we can always use the trick above to produce a sequence of transitions so that $w_i^A(k)= w_j^B(k)$.  And if the circuit is in this state there is a next availble transition following the original circuit

\end{document}