\documentclass{article}
\usepackage[utf8]{inputenc}
\usepackage[T1]{fontenc}
\usepackage{lmodern}

\title{Proof for normal operation of the duplicated scheme for asynchronous circuits}
\author{Zoey Zhou}
\date{\today}  
\usepackage{graphicx}
\graphicspath{ {c:/Users/cuizy/Downloads/} }

\usepackage{amsthm}
\newtheorem*{definition}{Definition}
\newtheorem*{claim}{Claim}
 
\begin{document}
\section{Normal operation analysis}
Definitions:  
\begin{definition}A state graph consists of <V, S, T> where $V$ is the ordered set of finite signals where $1\leq k \leq n$ indexes into $V$ and gives the signal $v_k$.  $S$ is the set of all states in the state graph (note that this is a subset of $2^n$) <-this might not be right math...
and s(k) denotes the (binary) value of the state at signal $v_k$.  T is the set of all transitions in the state graph $T \subseteq S X S$ and $(s_i, s_j) \in T$ denotes a transition from the state $s_i$ to $s_j$.  A transition pair $s_i$ and $s_j$ are under additional constraint 
$\exists k$ st $s_i(k)\neq s_j(k)$ and $s_i(l)=s_j(l)$ for all other $l \neq k$.\end{definition}
\newline
State graphs can be implemented as an asynchronous circuit where the signals of a state graph $s$ maps to wires in the circuit $w$. (one to one mapping?)  We can also map the set $S$ to a set $S_w$ of state of wires. The circuit has a set of transitions, for all $w_i \in S_w$, 
$(w_i, w_j)$ is a transition of the circuit if and only if $(s_i, s_j) \in T$.  An additional property of the asynchronous circuit is that one can view each wire as separate black boxes.  Inputs from the rest of the circuit (other $w_k$) feed into a logic block and outputs the wire value ()
An asynchronous circuit can be an implementation of a state graph (do I need to talk about state graph?)

We transform the circuit by making a duplicated copy of the logic components to each wire, and interconnecting the outputs of the logic components to two c-elements.  We want to show this circuit behaves similarly
compared to the original circuit under normal operation.  <need a formal def of similarly> Here we define similarly as given a trace in the original circuit, the same trace (with corresponding wires) can occur in the transformed circuit.  And if given a trace
in the transformed circuit, the same trace can occur in the original circuit. <define trace>  Of course this depends on the initial configuration of the transformed circuit.  But if we set up the initial correspondence correctly, we can show
certain properties in all of the traces that can occur.  
\newline
To show that the transformed circuit behaves similar to the original circuit, we have the following properties:
1)  the trace that occurs in A and B is a trace in original circuit (either trace or state)
2)  The correspondence property holds for the state in both A and B <define correspondence>
3)  if the same output wire in A and B are different, this indicates that one of the c-elements to A or B is excited. <do I need to define excited too>
\newline
We prove these properties through induction


\begin{definition}A state graph consists of <V, S, T> where V is the ordered set of signals where $1\leq k \leq n$ indexes into V and gives the signal $v_k$.  S is the set of all states in the state graph
and s(k) denotes the (binary) value of the state at signal $v_k$.  T is the set of all transitions in the state graph $T \subseteq S X S$ where $(s_i, s_j)$ denotes a transition from the state $s_i$ to $s_j$ 
with a single transition in some variable.\end{definition}
\begin{definition}Two state graphs $G_A$ and $G_B$ are similar if there exists a one to one mapping function f: $V_A \to V_B$.  So that for all $s \in S_A$, $\exists$ $s' \in S_B$ such that for all i in $n_A$,
$s(i)=s'(f(i))$, for simplicity we write $s\to s'$. And for all pairs of $(s,t) \in T_A$, $\exists (s',t') \in T_B$ and also for all pairs of $(s',t') \in T_B$, $\exists (s,t) \in T_A$ with $s\to s'$, $t\to t'$.\end{definition}

\begin{claim}Given some graph $G_A$ if we insert a signal that is a constant value (ie. 0 for all states) to form $G_B$ then $G_A$ is similar to $G_B$  \end{claim}
\begin{proof}
If we add the new signal to the beginning of the set of signals, $V_B={v_0} \cup V_A$ then the function f is just f(i)=i+1 for $1\leq i \leq n$.  And for all $s \in S_A$, we have $s\to s'$ where $s' \in S_B$.
Also the transitions in $G_B$ occur between (s',t') if and only if there is a transition (s,t) in $G_A$ with $s\to s'$, $t\to t'$.  Thus $G_A$ is similar to $G_B$
\end{proof}

Starting with a circuit that is a valid implementation of state graph $G$, we modify the circuit as follows, we add a $Restart_A$ signal, for every input A to the set or reset logic of some gate, we replace with 
$A$ or $Restart_A$.  The $Restart_A$ gate is implemented with a $0 \to 1$ restartable generalized C-element with no set signal, and a reset logic that is a copy of the reset logic of $A$.  If we allow a 0 signal 
being reset as staying in the 0 state then if we start at some state s0 from the original state graph $G$ and assign $Restart_A$ an initial value of 0 then the new state graph $G'$ we build from the new circuit starting
at state <0, s0>, the $Restart_A$ signal is a constant 0.  For all states $s \in S$ and corresponding state in the new state graph s'=<0, s>, then for every t such that $(s,t) \in T$, then $(s', t') \in T'$ where 
t'=<0, t> and since adding the new logic of $A$ or $Restart_A$ while $Restart_A$ is a constant 0 does not add new transitions to the new state graph that does not exist in $G$, this is an if and only if relation 
for all transitions starting from state s and s'.  Then by induction since there is an initial state s0 in $G$ and <0, s0> in $G'$, the rest of the states and transitions remain the same as in the original state graph $G$ 
with only an addition of the constant $Restart_A=0$ signal.  By our earlier claim, this new state graph $G'$ is similar to the original $G$.
\newline \newline
In the case of a stuck at fault in signal A, so that the $0\to1$ transition does not happen even when A is being set, then when enough time elapses the $Restart_A$ signal will go high. We build the resulting state graph from 
the circuit starting from this new state.  Since the new logic is $A$ or $Restart_A$, with $A$ now a constant 0 signal, and $Restart_A$ has the same reset logic as $A$, use the same argument as in the above paragraph 
but switch the signals so that we view $Restart_A$ as $A$ and $A$ as $Restart_A$, then the new state graph $G''$ that we build again has all of the states and trasitions as the original state graph $G$.  There is a slight 
difference in the mapping function f where f(1)= 1 (assuming A is the first signal in V) and f(i)=i+1 for $1< i \leq n$, then again it is inserting a constant signal to the original graph and thus $G''$ is similar to $G$.
\newline \newline
Thus this construction of a restartable circuit provides the same sequences of states as in the original circuit during normal operation as well as fault mode operation.  If we use the values of signals other than A, we cannot 
distinguish between the two circuits as it runs.


\end{document}